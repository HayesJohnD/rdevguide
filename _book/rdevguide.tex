\PassOptionsToPackage{unicode=true}{hyperref} % options for packages loaded elsewhere
\PassOptionsToPackage{hyphens}{url}
%
\documentclass[]{book}
\usepackage{lmodern}
\usepackage{amssymb,amsmath}
\usepackage{ifxetex,ifluatex}
\usepackage{fixltx2e} % provides \textsubscript
\ifnum 0\ifxetex 1\fi\ifluatex 1\fi=0 % if pdftex
  \usepackage[T1]{fontenc}
  \usepackage[utf8]{inputenc}
  \usepackage{textcomp} % provides euro and other symbols
\else % if luatex or xelatex
  \usepackage{unicode-math}
  \defaultfontfeatures{Ligatures=TeX,Scale=MatchLowercase}
\fi
% use upquote if available, for straight quotes in verbatim environments
\IfFileExists{upquote.sty}{\usepackage{upquote}}{}
% use microtype if available
\IfFileExists{microtype.sty}{%
\usepackage[]{microtype}
\UseMicrotypeSet[protrusion]{basicmath} % disable protrusion for tt fonts
}{}
\IfFileExists{parskip.sty}{%
\usepackage{parskip}
}{% else
\setlength{\parindent}{0pt}
\setlength{\parskip}{6pt plus 2pt minus 1pt}
}
\usepackage{hyperref}
\hypersetup{
            pdftitle={Contributing to R Core},
            pdfauthor={R Forwards},
            pdfborder={0 0 0},
            breaklinks=true}
\urlstyle{same}  % don't use monospace font for urls
\usepackage{longtable,booktabs}
% Fix footnotes in tables (requires footnote package)
\IfFileExists{footnote.sty}{\usepackage{footnote}\makesavenoteenv{longtable}}{}
\usepackage{graphicx,grffile}
\makeatletter
\def\maxwidth{\ifdim\Gin@nat@width>\linewidth\linewidth\else\Gin@nat@width\fi}
\def\maxheight{\ifdim\Gin@nat@height>\textheight\textheight\else\Gin@nat@height\fi}
\makeatother
% Scale images if necessary, so that they will not overflow the page
% margins by default, and it is still possible to overwrite the defaults
% using explicit options in \includegraphics[width, height, ...]{}
\setkeys{Gin}{width=\maxwidth,height=\maxheight,keepaspectratio}
\setlength{\emergencystretch}{3em}  % prevent overfull lines
\providecommand{\tightlist}{%
  \setlength{\itemsep}{0pt}\setlength{\parskip}{0pt}}
\setcounter{secnumdepth}{5}
% Redefines (sub)paragraphs to behave more like sections
\ifx\paragraph\undefined\else
\let\oldparagraph\paragraph
\renewcommand{\paragraph}[1]{\oldparagraph{#1}\mbox{}}
\fi
\ifx\subparagraph\undefined\else
\let\oldsubparagraph\subparagraph
\renewcommand{\subparagraph}[1]{\oldsubparagraph{#1}\mbox{}}
\fi

% set default figure placement to htbp
\makeatletter
\def\fps@figure{htbp}
\makeatother

\usepackage{booktabs}
\usepackage{amsthm}
\makeatletter
\def\thm@space@setup{%
  \thm@preskip=8pt plus 2pt minus 4pt
  \thm@postskip=\thm@preskip
}
\makeatother
\usepackage[]{natbib}
\bibliographystyle{apalike}

\title{Contributing to R Core}
\author{R Forwards}
\date{2020-07-13}

\begin{document}
\maketitle

{
\setcounter{tocdepth}{1}
\tableofcontents
}
\hypertarget{section}{%
\chapter*{}\label{section}}
\addcontentsline{toc}{chapter}{}

\hypertarget{initiatives-to-encourage-r-core-contributions}{%
\chapter{Initiatives to encourage R Core contributions}\label{initiatives-to-encourage-r-core-contributions}}

Many of these ideas are inspired by the work of other developer communities as I discuss here: \url{https://youtu.be/BbpkKzz71EY?t=1045}.

\hypertarget{documentation}{%
\chapter{Documentation}\label{documentation}}

Dev guide like devguide.python.org

\begin{itemize}
\tightlist
\item
  how to find issues in bugzilla
\item
  how to propose a patch
\item
  how to do patch reviews
\item
  document process of ``promotion'' to R core
\end{itemize}

c.f. \href{https://bookdown.org/lionel/contributing/}{Contributing to GNU R}, \href{https://www.r-project.org/bugs.html}{Bug Reporting}, \href{https://developer.r-project.org/Blog/public/2019/10/09/r-can-use-your-help-reviewing-bug-reports/index.html}{R can use your help reviewing bug reports}, \href{https://github.com/MichaelChirico/r-bugs}{Bugzilla mirror on GitHub}, \href{https://github.com/MichaelChirico/r-core-builder}{CI (Continuous Integration) for R-devel}, \href{https://github.com/jeroen/r-svn}{About
Mirror of R svn server with Github actions CI for testing patches}

\hypertarget{improve-use-of-bugzilla}{%
\chapter{Improve use of Bugzilla}\label{improve-use-of-bugzilla}}

\begin{itemize}
\item
  Use bugzilla for general issues tracking, not just bugs? (not really sure how this is done right now). Analogous to Python issue tracker. (FYI Python is moving issue tracker to GitHub.)
\item
  Establish intermediate roles e.g.~people able to triage issues on issue tracker? Analagous to becoming a Developer on Python issue tracker
\item
  Core devs/bugzilla veterans could offer mentoring/tutorial
\item
  Work with Forwards, R-Ladies, etc to reach potential contributors
\item
  Blog post by Tomas and Luke could become part of more comprehensive Developer Guide
\end{itemize}

\hypertarget{more-open-development}{%
\chapter{More open development}\label{more-open-development}}

\begin{itemize}
\item
  Move to git/GitHub
\item
  Requests for proposals on perceived deficiencies in R

  \begin{itemize}
  \tightlist
  \item
    community teams propose solutions
  \end{itemize}
\item
  R-ideas mailing list (analagous to python-ideas mailing list)
\item
  Analagous process to Python Enhancement Proposal (PEP)?

  \begin{itemize}
  \tightlist
  \item
    process would be described in the dev guide.
  \end{itemize}
\item
  Encourage more open developer-focused meetings

  \begin{itemize}
  \tightlist
  \item
    R core sessions at useR!
  \item
    link to developer-focused meetings e.g.~RIOT
  \item
    open attendance e.g.~at DSC
  \end{itemize}
\item
  Post about any initiatives on R blog

  \begin{itemize}
  \tightlist
  \item
    promote via Forwards, R-Ladies, MiR
  \end{itemize}
\end{itemize}

\hypertarget{identify-specific-areas-where-help-is-needed}{%
\chapter{Identify specific areas where help is needed}\label{identify-specific-areas-where-help-is-needed}}

\begin{itemize}
\tightlist
\item
  Help needed on recommended packages: MASS, survival
\item
  Testing pre-releases

  \begin{itemize}
  \tightlist
  \item
    write how-to?
  \item
    set up virtual machines for people to test on?
  \end{itemize}
\item
  Responding on R-devel/R-package-devel

  \begin{itemize}
  \tightlist
  \item
    introduce moderation?
  \end{itemize}
\end{itemize}

\hypertarget{mentoring}{%
\chapter{Mentoring}\label{mentoring}}

\begin{itemize}
\item
  Establish core-mentorship mailing list (separate from R-devel).
\item
  Use Zulip for more interactive discussion (and modern format)?

  \begin{itemize}
  \tightlist
  \item
    potential place for R-core/experienced contributors to offer office hours
  \end{itemize}
\item
  R-core/experienced contributors to offer bookable 1-to-1 office hours (e.g.~via Zoom)
\item
  Write a mentoring guide

  \begin{itemize}
  \tightlist
  \item
    c.f. \href{https://tinyurl.com/rladies-mentoring-guidelines}{R-Ladies mentoring guide}
  \end{itemize}
\item
  Run contributor tutorials

  \begin{itemize}
  \tightlist
  \item
    will need to be online
  \end{itemize}
\item
  Develop code of conduct

  \begin{itemize}
  \tightlist
  \item
    for contribution to R project?
  \item
    for mailing lists/Zulip
  \end{itemize}
\item
  Offer GSoC projects

  \begin{itemize}
  \tightlist
  \item
    scheme already established, could be used more by R Core
  \item
    or fund some of our own as Julia does: \url{https://julialang.org/jsoc/archive/}
  \end{itemize}
\item
  Similarly Google Season of Docs

  \begin{itemize}
  \tightlist
  \item
    \url{https://developers.google.com/season-of-docs}
  \end{itemize}
\item
  Establish mentored projects (diversity scholarships)

  \begin{itemize}
  \tightlist
  \item
    mentees invest 2-5 hours per week over 3 months
  \item
    expenses paid scholarship to useR! conference following year: is there some other incentive we can offer while useR! is online?
  \end{itemize}
\item
  Outreachy projects

  \begin{itemize}
  \tightlist
  \item
    paid internship, 40 hours per week over 3 months
  \item
    require mentor(s) to invest 5 hours per week
  \end{itemize}
\item
  Data umbrella sprints (similar to tidyverse developer days)

  \begin{itemize}
  \tightlist
  \item
    \url{https://www.dataumbrella.org/open-source/sprints}
  \end{itemize}
\end{itemize}

\hypertarget{funding}{%
\chapter{Funding}\label{funding}}

\begin{itemize}
\tightlist
\item
  Funding to support above efforts?

  \begin{itemize}
  \tightlist
  \item
    R Foundation likely to support some
  \item
    R Consortium + other sources to support FOSS projects?
  \end{itemize}
\end{itemize}

\hypertarget{openness-of-this-process-itself}{%
\chapter{Openness of this process itself}\label{openness-of-this-process-itself}}

\begin{itemize}
\tightlist
\item
  Get input from community on ideas via discuss forum like rOpensci?
\end{itemize}

\hypertarget{initiatives-to-encourage-cran-contributions}{%
\chapter{Initiatives to encourage CRAN contributions}\label{initiatives-to-encourage-cran-contributions}}

\begin{itemize}
\tightlist
\item
  create organization on GitHub for people to put/link their repos for feedback/review/help prior to CRAN release

  \begin{itemize}
  \tightlist
  \item
    e.g. ``seeking review'' tag?
  \item
    a bit like how Bioconductor does it but in a open, community distributed way?
  \item
    or more like ropensci model? This is more in-depth review, but could borrow some ideas:

    \begin{itemize}
    \tightlist
    \item
      match people with similar interests
    \item
      don't go back to same person too often
    \end{itemize}
  \end{itemize}
\item
  create community of on-rampers

  \begin{itemize}
  \tightlist
  \item
    sharing error messages from CRAN to create more user-friendly checklists, pointers to relevant parts of docs
  \end{itemize}
\item
  Zulip chat as alternative to R-package-devel to provide ad-hoc help
\item
  Step-by-step guide for people to review their own package before submission?

  \begin{itemize}
  \tightlist
  \item
    use some of tools from rOpensci?
  \end{itemize}
\item
  Chatbot like rOpenSci one helping with package review, to help submitters?
\end{itemize}

\bibliography{book.bib,packages.bib}

\end{document}
