\PassOptionsToPackage{unicode=true}{hyperref} % options for packages loaded elsewhere
\PassOptionsToPackage{hyphens}{url}
%
\documentclass[]{book}
\usepackage{lmodern}
\usepackage{amssymb,amsmath}
\usepackage{ifxetex,ifluatex}
\usepackage{fixltx2e} % provides \textsubscript
\ifnum 0\ifxetex 1\fi\ifluatex 1\fi=0 % if pdftex
  \usepackage[T1]{fontenc}
  \usepackage[utf8]{inputenc}
  \usepackage{textcomp} % provides euro and other symbols
\else % if luatex or xelatex
  \usepackage{unicode-math}
  \defaultfontfeatures{Ligatures=TeX,Scale=MatchLowercase}
\fi
% use upquote if available, for straight quotes in verbatim environments
\IfFileExists{upquote.sty}{\usepackage{upquote}}{}
% use microtype if available
\IfFileExists{microtype.sty}{%
\usepackage[]{microtype}
\UseMicrotypeSet[protrusion]{basicmath} % disable protrusion for tt fonts
}{}
\IfFileExists{parskip.sty}{%
\usepackage{parskip}
}{% else
\setlength{\parindent}{0pt}
\setlength{\parskip}{6pt plus 2pt minus 1pt}
}
\usepackage{hyperref}
\hypersetup{
            pdftitle={Contributing to R Core},
            pdfauthor={R Forwards},
            pdfborder={0 0 0},
            breaklinks=true}
\urlstyle{same}  % don't use monospace font for urls
\usepackage{longtable,booktabs}
% Fix footnotes in tables (requires footnote package)
\IfFileExists{footnote.sty}{\usepackage{footnote}\makesavenoteenv{longtable}}{}
\usepackage{graphicx,grffile}
\makeatletter
\def\maxwidth{\ifdim\Gin@nat@width>\linewidth\linewidth\else\Gin@nat@width\fi}
\def\maxheight{\ifdim\Gin@nat@height>\textheight\textheight\else\Gin@nat@height\fi}
\makeatother
% Scale images if necessary, so that they will not overflow the page
% margins by default, and it is still possible to overwrite the defaults
% using explicit options in \includegraphics[width, height, ...]{}
\setkeys{Gin}{width=\maxwidth,height=\maxheight,keepaspectratio}
\setlength{\emergencystretch}{3em}  % prevent overfull lines
\providecommand{\tightlist}{%
  \setlength{\itemsep}{0pt}\setlength{\parskip}{0pt}}
\setcounter{secnumdepth}{5}
% Redefines (sub)paragraphs to behave more like sections
\ifx\paragraph\undefined\else
\let\oldparagraph\paragraph
\renewcommand{\paragraph}[1]{\oldparagraph{#1}\mbox{}}
\fi
\ifx\subparagraph\undefined\else
\let\oldsubparagraph\subparagraph
\renewcommand{\subparagraph}[1]{\oldsubparagraph{#1}\mbox{}}
\fi

% set default figure placement to htbp
\makeatletter
\def\fps@figure{htbp}
\makeatother

\usepackage{booktabs}
\usepackage[]{natbib}
\bibliographystyle{apalike}

\title{Contributing to R Core}
\author{R Forwards}
\date{2020-07-27}

\begin{document}
\maketitle

{
\setcounter{tocdepth}{1}
\tableofcontents
}
\hypertarget{section}{%
\chapter*{}\label{section}}
\addcontentsline{toc}{chapter}{}

\hypertarget{r-core-developers-guide}{%
\chapter{R Core Developer's Guide}\label{r-core-developers-guide}}

This guide is heavily influenced by the \href{https://devguide.python.org/appendix/\#basics-for-contributors}{Python Developer Guide}, and is a comprehensive resource for contributing to R Core -- for both new and experienced contributors. It is maintained by {[}XXX{]}. We welcome your contributions to R Core!

\begin{note}

This box denotes a tip for the reader.

\end{note}

\hypertarget{quick-reference}{%
\section{Quick Reference}\label{quick-reference}}

\hypertarget{quick-links}{%
\section{Quick Links}\label{quick-links}}

\hypertarget{status-of-r-core-branches}{%
\section{Status of R Core Branches}\label{status-of-r-core-branches}}

\hypertarget{contributing}{%
\section{Contributing}\label{contributing}}

\hypertarget{proposing-changes-to-r-core}{%
\section{Proposing Changes to R Core}\label{proposing-changes-to-r-core}}

\hypertarget{other-interpreter-implementations}{%
\section{Other Interpreter Implementations}\label{other-interpreter-implementations}}

\hypertarget{key-resources}{%
\section{Key Resources}\label{key-resources}}

\hypertarget{additional-resources}{%
\section{Additional Resources}\label{additional-resources}}

\hypertarget{code-of-conduct}{%
\section{Code of Conduct}\label{code-of-conduct}}

\hypertarget{diving-in}{%
\chapter{Diving In}\label{diving-in}}

Now let's talk details.

\hypertarget{technical-details}{%
\chapter{Technical Details}\label{technical-details}}

Now I'll teach you some crazy math, but I need to work it out first\ldots{}

\bibliography{book.bib}

\end{document}
